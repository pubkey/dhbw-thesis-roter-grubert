\section{Kapitel 1}


\subsection{Management Summary}

TODO

\subsection{Motivation}

\begin{quote}
"`Prozessorientierung ist seit Beginn der 90er Jahre als eine unverzichtbare 
Maxime der Unternehmensgestaltung akzeptiert. In den letzten Jahren 
haben viele Unternehmen Maßnahmen
 zur verstärkten Ausrichtung an ihren Geschäftsprozessen initiiert."' 
\footcite[S.182]{prozessmanagement:leitfaden}
\end{quote}


Aussagen wie diese, zeigen dass die Prozessorientierte
Geschäftsprozessmodelierung mittlerweile fest in allen Unternehmen angekommen ist.
Umso wichtiger ist es also, die Unterschiede und Feinheiten verschiedener Tools,
 Prozesse und Anwendungsszenarien zu kennen.

\subsection{Ziel der Arbeit}

Die vorliegende Arbeit soll dem Leser einen vollständigen Überblick über die
Kernbereiche der Prozessmodelierung geben.



\section{Geschäftsprozesse}

\subsection{Prozess}


Ein „Prozess“, vom lateinischen „processus“ („Fortgang, Fortschreiten“) ist von
der Wiederholung bereits existierender Vorgänge geprägt.

Es handelt sich dabei also um eine sich äufig wiederholte, eher sequentielle
Verkettung von Aktivitäten, wobei die Ausgangslage sowie das
angestrebte Ergebnis definiert und die erforderlichen Maßnahmen
kategorisiert bzw. spezifiziert sind. Dabei besetehen stets nur
unbedeutende Unsicherheiten in der Zielerreichung zum Beispiel "`Beschaffung
eines Zulieferteils"'.

\subsection{Definition und Management von Geschäftsprozessen}


Ein Geschäftsprozess besteht aus der wiederkehrenden Abfolge von logischen,
zeitlich zusammengehörigen und inhaltlich abgeschlossenen Aktivitäten, dessen 
Durchführung als Ziel die Wünsche der Kunden, die ein Unternehmen besitzt, zu befriedigen trägt.
Eine höhere Kundenzufriedenheit bedeutet ebenfalls eine höhere Wertschöpfung. 
Mit einem bestimmten Input und bestimmtem Ressourceneinsatz, 
entsteht ein Output, der an einen Empfänger geht. 
Dieser Empfänger kann ein Kunde, ein Lieferant oder eine innerbetriebliche Stelle sein. 
Handelt es sich tatsächlich um einen innerbetrieblichen Prozess, 
so kann davon ausgegangen werden, dass dieser organisatorisch dauerhaft geregelt wird.
\footcite[Vgl.][ ]{prozess:db}

Zusätzlich wird unterschieden zwischen Leistungs-, Unterstützungs- und
Führungsprozessen den sogenannten Prozessarten. Prozesse werden analysiert, 
woraus eine Bewertung ihrer Funktionalität und Effektivität entsteht. 
Ziel der Prozessanalyse ist vor allem die Schaffung einer Transparenz als 
Voraussetzung für eine bestmögliche Prozesssteuerung, dem „Geschäftsprozessmanagement“. 
Eine einfache Wortanalyse ergibt, dass sich das Geschäftsprozessmanagement 
mit der Verwaltung von Geschäftsprozessen beschäftigt.
\footcite[S.13]{lehmann}

Dazu zählen insbesondere folgende Aspekte:
Identifikation, Planung, Dokumentation,  Gewichtung, Verbesserung,
Steuerung, Kontrolle und Organisation von Geschäftsprozessen.
Zusammengefasst ist das Hauptziel und damit allgemeine Anforderung in Unternehmen, 
alle Prozessaktivitäten und Prozesse möglichst effizient auszuführen. 
Die Herausforderung für Unternehmen besteht darin, dass diese 
möglichst unterbrechungs- und fehlerfrei ablaufen. 
Wenn die Prozesse und Unterprozesse einmal identifiziert worden sind, 
erfolgt die Beachtung der oben genannten Aspekte in Form einer Modellierung. 
Ein Prozessmodell dient dazu die kombinierte, überschneidungsfreie 
und lückenlose Struktur zusammenhängender Prozesse in einem Unternehmen darzustellen. 
Hierzu gibt es verschiedene Möglichkeiten. 
Im einfachsten Fall werden textuelle oder tabellarische Beschreibungen verwendet. 
Häufig werden Präsentations- oder Grafikprogramme genutzt, um einfache Ablaufdiagramme 
zu erstellen. Sie bestehen meist aus Kästchen und Pfeilen, wobei keiner 
bestimmten Methodik gefolgt wird. Zur genauen Darstellung 
komplexerer Prozesse mit allen relevanten Aspekten, wie Verzweigungsregeln, 
Ereignissen, ausführenden Organisationseinheiten, Datenflüssen usw., 
genügt eine grobe Modellierung nicht. Hierfür werden geeignete Notationen benötigt. 
Mit welchen Symbolen die verschiedenen Elemente von Prozessen dargestellt werden, 
was sie genau bedeuten und wie sie miteinander kombiniert werden können, 
wird durch die Notation festgelegt. (Prozessmodellierung)
Welche Merkmale in Prozessmodellen abgebildet werden, hängt vom 
Modellierungszweck und der verwendeten Modellierungssprache ab. 
Die Regeln zur Modellierung von Geschäftsprozessen werden in folgenden Kapiteln deutlich.


\section{Prozessmodellierung und Notationen}

Im Zuge der Modellierung hat man es mit Zeichen unterschiedlichster Art zu tun.
Prozessmodelle können in Form von Texten, Tabellen oder Grafiken dargestellt werden. 
Üblicherweise wird eine Modellierungssprache verwendet, 
die eine Notation zur Abbildung von Geschäftsprozessen zur Verfügung stellt. 
In dieser Arbeit wird insbesondere mit der Notation BPMN gearbeitet.




\subsection{Ereignisgesteuerte Prozesskette und Business Process Model and
Notation}


EPK ist als Abkürzung einer ereignisgesteuerten Prozesskette zu verstehen und
ist für die detaillierte Modellierung und Veranschaulichung
von Geschäftsprozessen und Prozesselementen gut
geeignet.\footcite[Vgl.][]{lehmann}
 

Diese Notationsform ist 1992 unter der Leitung von
Scheer\footcite[Vgl.][]{scheer} entwickelt worden, somit wird die Bezeichnung
Sheer-Notation oft verwendet.
EPKs sind das Hauptdarstellungsmittel in Architecture of Integrated 
Information Systems (ARIS), darunter Ereignisse, 
Funktionen und Verknüpfungsoperatoren. Es gibt ebenfalls eine erweiterte EPK, 
die weitere Merkmale, wie bspw. Organisationseinheiten, Rollen von Mitarbeitern, 
sowie Datenbestände bzw. Informationssysteme, bietet. 
Da die Entwicklung und Pflege sehr umfangreich werden können ist die 
Nutzung von Softwarewerkzeugen notwendig. 
ARIS unterstützt Unternehmen bei der Modellierung, Analyse und Optimierung von Prozessen.
Das Grundproblem, nämlich der steigende Wettbewerbsdruck bezüglich der Zeit, 
Kosten und Qualität verlangt effiziente und effektive Organisationsformen. 
Unter einer Prozessorganisation ist nun eine Organisationsform zu verstehen, 
bei der die Strukturierung von organisatorischen Einheiten, 
insbesondere  Prozessteams bzw. Funktionsbereiche, 
den Kern- und Unterstützungsprozessen folgt.






\subsection{Kapitel 1.1\ldots die AUF JEDENFALL MEHRZEILIG\\IST}

Ldforem ips um do lor sit amet, consetetur sadipscing elitr, \label{Referenz}
sed diam  nonumy eirmod tempor invidunt ut labore et dolore magna aliquyam erat,
sed diam vddoluptsssua. At vero eos et accusam et justo duo dolores et ea rebum.
Stet clita kasd gubergren, no sea takimata sdner anctus est Lorem ipsum dolor sit amet. Lorem ipsum dolor sit amet, consetetur sadipscing elitr, sed diam nonumy eirmod tempor invidunt ut labore et dolore magna aliquyam erat, sed diam voluptua. At vero eos et accusam et justo duo dolores et ea rebum. Stet clita kasd gubergren, no sea takimata sanctus est Lorem ipsum dolor sit amet.
\footcite[Vgl.][Experto.de, Artikel über das und jenes]{praxishandbuch:bpmn2}
\subsubsection{Beispieltext}
Lorem ipsum dolor sit amet, co nsetetur sadipscing elitr, sed diam
nonumy eirmod tempor invidunt ut labore et dolore magna aliquyam erat, sed diam voluptua. At vero eos et accusam et justo duo dolores et ea rebum. Stet clita kasd gubergren, no sea takimata sanctus est Lorem ipsum dolor sit amet. Lorem ipsum dolor sit amet, consetetur sadipscing elitr, sed diam nonumy eirmod tempor invidunt ut labore et dolore magna aliquyam erat, sed diam voluptua. At vero eos et accusam et justo duo dolores et ea rebum. Stet clita kasd gubergren, no sea takimata sanctus est Lorem ipsum dolor sit amet.

%erzwinge Seitenumbruch
\clearpage

\subsection{Beispielbilder}
Lorem ipsum dolor sit amet, consetetur  sdf sadipscing elitr, sed diam nonumy eirmod tempor invidunt ut labore et dolore magna aliquyam erat, sed diam voluptua. At vero eos et accusam et justo duo dolores et ea rebum. Stet

\begin{center}
	\includegraphics[width=5cm]{images/company_logo.png}
	\captionof{figure}{Beispielbild - Quelle: Internet}
\end{center}

Lorem ipsum dolor sit amet, consetetur sadipscing elitr, sed diam nonumy eirmod tempor invidunt ut labore et dolore magna aliquyam erat, sed diam voluptua. At vero eos et accusam et justo duo dolores et ea rebum. Stet
\begin{center}
	\includegraphics[width=10cm]{images/institute_logo.png}
	\captionof{figure}{Beispielbild2 - Quelle: Internet}
\end{center}


\clearpage

\subsection{Beispieltabellen}
Lorem ipsum dolor sit amet, consetetur sadipscing elitr, sed diam.\\\\
 nonumyeirmod tempor invidunt ut labore et dolore magna aliquyam erat, sed diam voluptua. At vero eos et accusam et justo duo dolores et ea rebum. Stet


\begin{table}[h]
\begin{tabularx}{\textwidth}{|l|c|X|r|}
	\hline
	\textbf{Spalte 1} (linksbündig) & \textbf{Spalte 2} (center) &
	blabla & \textbf{Spalte1} (rechts) \\
	\hline
	\hline
	inhalt1 & foobar & inhalt 3 & aaaa \\
	\hline
	inhalt1 & inhalt 2 & inhalt 3 & bbbb \\
	\hline
	inhalt1 & inhalt 2 & inhalt 3 & ccccc \\
	\hline
\end{tabularx}
\caption{Beispieltabelle 1}
\end{table}


\section{Die Kunst des Dönermachens}

Beispielzitat bei mehr als 2 Zeilen:
\begin{quote}
"`Döner macht schö2ner über mehr als 5 Zeilen\\
Macht Ali dir auch einen Döner \\
ohne Extra Soße \\
aber den Käse vergisst er trozdem\\
ENDE."' 
\footcite[Wörtlich übernommen von Ali]{praxishandbuch:bpmn2}
\end{quote}
Verweis innerhalb des Dokuments:\footcite[Vgl. \ref{Referenz} auf Seite
\pageref{Referenz} ]{testarticle}


\begin{figure}[H]
\begin{minipage}{\linewidth}
\begin{center}
\fbox{
\centering
\includegraphics[width=10cm]{images/company_logo.png}
\caption{
Schauerleute im Hamburger Hafen
\newline
Quelle: {\protect\url{
https://de.wikipedia.org/wiki/Hafenarbeiter#/media/File:Dockers\%27_work_difference.jpg
}}\newline
Aufruf: 13.10.2015
}
}
\end{center}
\end{minipage}
\end{figure}

\clearpage
\section{Anhang}

Lorem ipsum\ldots
